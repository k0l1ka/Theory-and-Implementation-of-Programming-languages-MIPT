\documentclass[a4paper,12pt]{article} % тип документа

% report, book

%  Русский язык

\usepackage[T2A]{fontenc}			% кодировка
\usepackage[utf8]{inputenc}			% кодировка исходного текста
\usepackage[english,russian]{babel}	% локализация и переносы


% Математика
\usepackage{amsmath,amsfonts,amssymb,amsthm,mathtools} 


\usepackage{wasysym}
% мои добьавки-----------------------------------------
\usepackage{ dsfont }
\usepackage{listings}

%мое----------------------------------------------------
%Заговолок
\author{Попов Николай}
\title{Домашнее задание 1\\ по алгебре \LaTeX{}}

%--------------------
\usepackage{tikz}












\begin{document} 
\section*{Задача 1}


\begin{center}
\begin{tikzpicture}[scale=0.2]
\tikzstyle{every node}+=[inner sep=0pt]
\draw [black] (10.1,-12) circle (3);
\draw (10.1,-12) node {$1$};
\draw [black] (25.5,-12) circle (3);
\draw (25.5,-12) node {$2$};
\draw [black] (43.2,-4.6) circle (3);
\draw (43.2,-4.6) node {$3$};
\draw [black] (42.3,-20.8) circle (3);
\draw (42.3,-20.8) node {$4$};
\draw [black] (60.1,-13.7) circle (3);
\draw (60.1,-13.7) node {$5$};
\draw [black] (75.3,-13.7) circle (3);
\draw (75.3,-13.7) node {$6$};
\draw [black] (75.3,-13.7) circle (2.4);
\draw [black] (0.6,-11.4) -- (7.11,-11.81);
\fill [black] (7.11,-11.81) -- (6.34,-11.26) -- (6.28,-12.26);
\draw [black] (13.1,-12) -- (22.5,-12);
\fill [black] (22.5,-12) -- (21.7,-11.5) -- (21.7,-12.5);
\draw (17.8,-12.5) node [below] {$a$};
\draw [black] (28.27,-10.84) -- (40.43,-5.76);
\fill [black] (40.43,-5.76) -- (39.5,-5.6) -- (39.89,-6.53);
\draw (35.41,-8.82) node [below] {$a$};
\draw [black] (28.16,-13.39) -- (39.64,-19.41);
\fill [black] (39.64,-19.41) -- (39.17,-18.59) -- (38.7,-19.48);
\draw (32.76,-16.9) node [below] {$b$};
\draw [black] (23.475,-9.803) arc (250.39653:-37.60347:2.25);
\draw (22.65,-4.97) node [above] {$b$};
\fill [black] (26.01,-9.06) -- (26.75,-8.47) -- (25.81,-8.13);
\draw [black] (45.84,-6.02) -- (57.46,-12.28);
\fill [black] (57.46,-12.28) -- (56.99,-11.46) -- (56.52,-12.34);
\draw (50.51,-9.65) node [below] {$b$};
\draw [black] (45.09,-19.69) -- (57.31,-14.81);
\fill [black] (57.31,-14.81) -- (56.39,-14.64) -- (56.76,-15.57);
\draw (52.3,-17.77) node [below] {$b$};
\draw [black] (58.777,-11.02) arc (234:-54:2.25);
\draw (60.1,-6.45) node [above] {$a$};
\fill [black] (61.42,-11.02) -- (62.3,-10.67) -- (61.49,-10.08);
\draw [black] (63.1,-13.7) -- (72.3,-13.7);
\fill [black] (72.3,-13.7) -- (71.5,-13.2) -- (71.5,-14.2);
\draw (67.7,-14.2) node [below] {$b$};
\end{tikzpicture}
\end{center}





$R^0$\\
\[\begin{vmatrix}
\hspace{15pt}&\hspace{10pt}1\hspace{10pt}&\hspace{10pt}2\hspace{10pt}&\hspace{10pt}3\hspace{10pt}&\hspace{10pt}4\hspace{10pt}&\hspace{10pt}5\hspace{10pt}&\hspace{10pt}6\hspace{10pt}\\
1& & a & \\
2& & b & a & b\\
3& & & & & b\\
4& & & & & b\\
5& & & & & a & b\\
6& \\
\end{vmatrix}
\]

$R^1$\\
\[\begin{vmatrix}
\hspace{15pt}&\hspace{10pt}1\hspace{10pt}&\hspace{10pt}2\hspace{10pt}&\hspace{10pt}3\hspace{10pt}&\hspace{10pt}4\hspace{10pt}&\hspace{10pt}5\hspace{10pt}&\hspace{10pt}6\hspace{10pt}\\
1& & a & \\
2& & b & a & b\\
3& & & & & b\\
4& & & & & b\\
5& & & & & a & b\\
6& \\
\end{vmatrix}
\]


$R^2$\\

\[\begin{vmatrix}
\hspace{15pt}&\hspace{10pt}1\hspace{10pt}&\hspace{10pt}2\hspace{10pt}&\hspace{10pt}3\hspace{10pt}&\hspace{10pt}4\hspace{10pt}&\hspace{10pt}5\hspace{10pt}&\hspace{10pt}6\hspace{10pt}\\
1& & a|ab^*b & ab^*a & ab^*b \\
2& & b|bb^*b & a|bb^*a & b|bb^*b\\
3& & & & & b\\
4& & & & & b\\
5& & & & & a & b\\
6& \\
\end{vmatrix}
\]


$R^3$\\

\[\begin{vmatrix}
\hspace{15pt}&\hspace{10pt}1\hspace{10pt}&\hspace{10pt}2\hspace{10pt}&\hspace{10pt}3\hspace{10pt}&\hspace{10pt}4\hspace{10pt}&\hspace{10pt}5\hspace{10pt}&\hspace{10pt}6\hspace{10pt}\\
1& & a|ab^*b & ab^*a & ab^*b & ab^*ab\\
2& & b|bb^*b & a|bb^*a & b|bb^*b & (a|bb^*a)b\\
3& & & & & b\\
4& & & & & b\\
5& & & & & a & b\\
6& \\
\end{vmatrix}
\]



$R^4$\\

\[\begin{vmatrix}
\hspace{15pt}&\hspace{10pt}1\hspace{10pt}&\hspace{10pt}2\hspace{10pt}&\hspace{10pt}3\hspace{10pt}&\hspace{10pt}4\hspace{10pt}&\hspace{10pt}5\hspace{10pt}&\hspace{10pt}6\hspace{10pt}\\
1& & a|ab^*b & ab^*a & ab^*b & ab^*ab|ab^*bb\\
2& & b|bb^*b & a|bb^*a & b|bb^*b & (a|bb^*a)b|(b|bb^*b)b\\
3& & & & & b\\
4& & & & & b\\
5& & & & & a & b\\
6& \\
\end{vmatrix}
\]


$R^5$\\

\[\begin{vmatrix}
\hspace{5pt}&\hspace{5pt}1\hspace{5pt}&\hspace{5pt}2\hspace{5pt}&\hspace{5pt}3\hspace{5pt}&\hspace{5pt}4\hspace{5pt}&\hspace{5pt}5\hspace{5pt}&\hspace{5pt}6\hspace{5pt}\\
1& & a|ab^*b & ab^*a & ab^*b & ab^*ab|ab^*bb|(ab^*ab|ab^*bb)a^*a & (ab^*ab|ab^*bb)a^*b\\
2& & b|bb^*b & a|bb^*a & b|bb^*b & (a|bb^*a)b|(b|bb^*b)b|((a|bb^*a)b|(b|bb^*b)b)ba^*a & ((a|bb^*a)b|(b|bb^*b)b)ba^*b\\
3& & & & & b|ba^*a & ba^*b\\
4& & & & & b|ba^*a & ba^*b\\
5& & & & & a|aa^*a & b|aa^*b\\
6& \\
\end{vmatrix}
\]

В финальной таблице $R^6$ изменится только ячейка (1,6), поэтому ее напишу отдельно:\\
\[
R^6_{16}=(ab^*ab|ab^*bb)a^*b|(
ab^*ab|ab^*bb|(ab^*ab|ab^*bb)a^*a)
(a|aa^*a)^*(b|aa^*b)
\]
Получил РВ, эквивалентное ДКА.\\

\section*{Задача 2}

Т.к. ДКА полный, то он прочтет до конца любое слово из $T^*$.
Рассмотрим два произвольных слова u и v из одного класса эквивалентности по L. Предположим, что при их прочтении посредством ДКА, автомат остановился в разных состояниях. Теперь "допишем"
к каждому из этих двух слов произвольную цепочку символов $t \in T^*$. Из определения эквивалентности получаем, что слова ut и vt одновременно принадлежат или не принадлежат языку L. Значит, при прочтении этих слов автоматом, он либо остановится в конечных сосотояниях $q_f1,q_f2$, либо остановится в неконечных состояниях $q_nf1,q_nf2$для обоих слов ut и vt. Учитывая, что автомат минимальный(т.е. в нем нет никаких двух состояний, из которых по одному и тому символу переход происходит в одно состояние), получаем, что цепочка t прочитывается в нем только один раз (т.е. последовательность состояний и ребер, проход по которым прочитывает цепочку t, в автомате единственна). Из этого следует, что после прочтения u и v и перед прочтением t минимальный автомат должен находиться в одном и том же состоянии, а наше предположение противоречит минимальности автомата.\\



\section*{Задача 3}
Разделим все слова z из $T^*$ 
по их длине:\\
1)Первая группа - \{ $z \in T^*: |z| \geq$ |w|\}. Они делятся на |w|+2 класса по языку L = PreSuf(w): в первом классе $C_1$ слова, не начинающиеся с w. При дописывании к ним чего угодно они попадают в $\tilde{L}$(дополнение L). Остальные слова, т.е. начинающиеся с w, разделяться |w|+1 классов - $\{Q_i\}$, аналогично языку Suf(w) с семинара.\\
2)Вторая группа - слова \{ $z \in T^*: |z| < $ |w|\}. Слова отсюда попадут в класс $C_1$, если не являются префиксом w.  
Остается рассмотреть только префиксы w. Каждый префикс образует собвественный (еще не рассмотренный) класс $P_i$, т.к., очевидно, что префиксы не попадают в класс $C_1$ и ни один из "префиксных" классов не совпадает ни с одним классом из$\{Q_i\}$, т.к. чтобы получить слово из языка префикс нужно "дописать" до слова w, а слово из $\{Q_i\}$ - не нужно. Префиксов всего |w|(с нулевым и не считая слова w).\\
Таким образом, получено 2*|w|+2 классов экв-ти. Т.к. кол-во классов конечно, то язык регулярен. Т.к. число классов экв-ти регулярного языка равно числу состояний в min ПДКА, то в min ПДКА, распознающем PreSuf(w), будет 2*|w|+2 состояний.\\


\section*{Задача 4}
Классов эквивалентости для языка правильных скобочных последовательностей L не конечное число. Это следует из того, что язык нерегулярен, что несложно получить от противного по лемме о накачке для правильных скобочных последовательностей вида \{$(^n)^n, n\geq 0$\}. \\

Слова входящие в L, входят в один класс $C_1$, т.к. дописывание к ним всяких слов из L не выыводит их из L, а дописывание  слова из $\tilde{L}$ - выводит.\\

Теперь изучим $\tilde{L}$. Если слово имеет вид $(S_1(S_2...(S_n$, где $S_i$- пустая строка или правильная скобочная последовательность, то оно попадает в класс, взаимооднознозначно соответсвующий числу открывающих скобок в нем, не считая $S_i$. Тогда дописав всем таким словам любое слово вида $S_a)S_b)...S_p)$, в котором то же число закрывающих скобок, не считая (не считая $S_i$), попадаем в L, иначе (т.е. когда другое число скобок ")" или среди $S_i$ есть неправильные последовательности) - попадаем в $\tilde{L}$.\\

 Еще в один новый класс попадают все слова вида $(S_1(S_2...(S_n$, где хотя бы одна из подпоследовательностей $S_i$ - неправильная (при дописании любого слова они попадают в $\tilde{L}$).\\
 
 
\section*{Задача 5}
Мое имя - nick=w, |w|=4\\

Опираясь на задачи 2 и 3 получаем, что состояний 10.\\
состояние - класс:\\
1 - e(пустое слово)\\
2 - n\\
3 - ni\\
4 - nic\\
5 - $\{x|overlap(x,w)=4\}$\\
6 - $\{x|overlap(x,w)=3\}$\\
7 - $\{x|overlap(x,w)=2\}$\\
8 - $\{x|overlap(x,w)=1\}$\\
9 - $\{x|overlap(x,w)=0\}$\\
10 - слова с началом  $\tilde{w}$(не w)
\\


\begin{center}
\begin{tikzpicture}[scale=0.2]
\tikzstyle{every node}+=[inner sep=0pt]
\draw [black] (5.2,-4.9) circle (3);
\draw (5.2,-4.9) node {$1$};
\draw [black] (14.9,-3.8) circle (3);
\draw (14.9,-3.8) node {$2$};
\draw [black] (7.5,-20.6) circle (3);
\draw (7.5,-20.6) node {$10$};
\draw [black] (24.3,-3.2) circle (3);
\draw (24.3,-3.2) node {$3$};
\draw [black] (34,-3.8) circle (3);
\draw (34,-3.8) node {$4$};
\draw [black] (57.9,-7.8) circle (3);
\draw (57.9,-7.8) node {$5$};
\draw [black] (57.9,-7.8) circle (2.4);
\draw [black] (53.7,-16.7) circle (3);
\draw (53.7,-16.7) node {$6$};
\draw [black] (48,-26.2) circle (3);
\draw (48,-26.2) node {$7$};
\draw [black] (67.8,-32) circle (3);
\draw (67.8,-32) node {$8$};
\draw [black] (62.6,-49.7) circle (3);
\draw (62.6,-49.7) node {$9$};
\draw [black] (0.5,-0.3) -- (3.06,-2.8);
\fill [black] (3.06,-2.8) -- (2.83,-1.88) -- (2.13,-2.6);
\draw [black] (8.18,-4.56) -- (11.92,-4.14);
\fill [black] (11.92,-4.14) -- (11.07,-3.73) -- (11.18,-4.72);
\draw (10.32,-4.95) node [below] {$n$};
\draw [black] (6.237,-17.881) arc (-158.85331:-184.47796:22.82);
\fill [black] (6.24,-17.88) -- (6.41,-16.95) -- (5.48,-17.32);
\draw (4.25,-13.14) node [left] {$c,i,k$};
\draw [black] (17.89,-3.61) -- (21.31,-3.39);
\fill [black] (21.31,-3.39) -- (20.48,-2.94) -- (20.54,-3.94);
\draw (19.67,-4.05) node [below] {$i$};
\draw [black] (27.29,-3.39) -- (31.01,-3.61);
\fill [black] (31.01,-3.61) -- (30.24,-3.07) -- (30.18,-4.06);
\draw (29.07,-4.06) node [below] {$c$};
\draw [black] (36.927,-3.15) arc (99.52351:61.4741:28.641);
\fill [black] (55.34,-6.23) -- (54.88,-5.41) -- (54.4,-6.29);
\draw (46.88,-2.54) node [above] {$k$};
\draw [black] (13.69,-6.55) -- (8.71,-17.85);
\fill [black] (8.71,-17.85) -- (9.49,-17.32) -- (8.57,-16.92);
\draw (10.47,-11.22) node [left] {$n,c,k$};
\draw [black] (22.671,-5.718) arc (-34.45875:-53.53107:55.229);
\fill [black] (9.96,-18.88) -- (10.9,-18.81) -- (10.31,-18.01);
\draw (17.39,-14.3) node [right] {$n,i,k$};
\draw [black] (32.645,-6.475) arc (-29.91735:-85.33634:28.197);
\fill [black] (10.5,-20.52) -- (11.34,-20.95) -- (11.25,-19.95);
\draw (25.9,-16.73) node [below] {$n,i,c$};
\draw [black] (54.224,-13.752) arc (163.49054:145.98326:13.349);
\fill [black] (55.96,-10.08) -- (55.1,-10.46) -- (55.92,-11.02);
\draw (54.24,-10.8) node [left] {$k$};
\draw [black] (48.565,-23.261) arc (162.03248:136.04001:12.135);
\fill [black] (51.37,-18.58) -- (50.46,-18.81) -- (51.18,-19.5);
\draw (49.06,-19.5) node [left] {$c$};
\draw [black] (65.601,-34.027) arc (-54.97463:-157.67908:11.236);
\fill [black] (48.76,-29.09) -- (48.6,-30.02) -- (49.52,-29.64);
\draw (55.34,-36.15) node [below] {$i$};
\draw [black] (63.45,-46.82) -- (66.95,-34.88);
\fill [black] (66.95,-34.88) -- (66.25,-35.5) -- (67.21,-35.79);
\draw (65.97,-41.44) node [right] {$n$};
\draw [black] (59.859,-50.914) arc (-69.91521:-277.28435:22.624);
\fill [black] (59.86,-50.91) -- (58.94,-50.72) -- (59.28,-51.66);
\draw (28.95,-32.3) node [left] {$c,i,k$};
\draw [black] (60.603,-6.514) arc (108.95328:-64.45523:13.262);
\fill [black] (70.63,-31.02) -- (71.57,-31.13) -- (71.14,-30.23);
\draw (77.93,-13.13) node [right] {$n$};
\draw [black] (55.756,-14.529) arc (128.47878:-43.15339:10.635);
\fill [black] (70.13,-30.13) -- (71.04,-29.89) -- (70.31,-29.2);
\draw (70.73,-14.18) node [right] {$n$};
\draw [black] (59.671,-50.332) arc (-82.85316:-246.96006:17.101);
\fill [black] (59.67,-50.33) -- (58.82,-49.94) -- (58.94,-50.93);
\draw (40.27,-38.33) node [left] {$c,i$};
\draw [black] (50.26,-24.24) arc (123.39418:23.95212:11.401);
\fill [black] (66.95,-29.13) -- (67.09,-28.2) -- (66.17,-28.6);
\draw (60.69,-22.26) node [above] {$n$};
\draw [black] (59.707,-48.917) arc (-109.73269:-186.56387:18.735);
\fill [black] (59.71,-48.92) -- (59.12,-48.18) -- (58.79,-49.12);
\draw (49.49,-42.46) node [left] {$k,i$};
\draw [black] (70.443,-33.392) arc (52.90713:-85.65116:9.23);
\fill [black] (65.58,-49.96) -- (66.41,-50.4) -- (66.34,-49.4);
\draw (74.5,-43.94) node [right] {$c,k$};
\draw [black] (63.923,-52.38) arc (54:-234:2.25);
\draw (62.6,-56.95) node [below] {$c,i,k$};
\fill [black] (61.28,-52.38) -- (60.4,-52.73) -- (61.21,-53.32);
\draw [black] (64.889,-32.674) arc (310.75948:22.75948:2.25);
\draw (60.47,-30.04) node [left] {$n$};
\fill [black] (65.49,-30.1) -- (65.35,-29.17) -- (64.59,-29.82);
\end{tikzpicture}
\end{center}

Все слова из языка будут приняты автоматом, т.к. по построению финальными состояниями в нем являются состояния, эквивалентные классам К, которые являются подмножествами самого языка. Причем каждое слово из языка принадлежит какому-нибудь такому классу К, т.к. все $T^*$ разбито на классы. 
Т.е. слово при чтении слова из языка автомат обязательно пройдет сотояния 1-2-3-4-5 и вернется в него по одному из путей (9-8-7-6-5),(8-7-6-5),(7-6-5),(6-5), если после w в этом слове есть еще символы, т.к. слово заканчивается на w. \\

Если слово принято автоматом, то оно принадлежит языку, т.к. если при чтении слова автомат остановился в конечном состоянии (5), отображающем соответсвующий класс, который является подмножеством языка.\\


\end{document}
