\documentclass[a4paper,12pt]{article} % тип документа

% report, book

%  Русский язык

\usepackage[T2A]{fontenc}			% кодировка
\usepackage[utf8]{inputenc}			% кодировка исходного текста
\usepackage[english,russian]{babel}	% локализация и переносы


% Математика
\usepackage{amsmath,amsfonts,amssymb,amsthm,mathtools} 


\usepackage{wasysym}
% мои добьавки-----------------------------------------
\usepackage{ dsfont }
\usepackage{listings}

%мое----------------------------------------------------
%Заговолок
\author{Попов Николай}
\title{Домашнее задание 1\\ по алгебре \LaTeX{}}

\begin{document} % начало документа


\newpage 

\section*{Задача 1}
$G: N=\{S, K\} $ $T=\{a, b\}$  $S=S$  \\

$P:\{S\rightarrow aaKb, K\rightarrow aK, K\rightarrow aKb, K\rightarrow \epsilon \}$\\

Док-во: индукцией по числу шагов покажем что на n-ом шаге грам-ка G порождает элемент из \[ \{a^{n+2}Kb, a^{2+n}Kb^{1+n}, a^{n+1}b, a^{1+n}b^{n}, a^{2+x+y}Kb^{1+y}, a^{1+x+y}b^{y}, a^{1+x+y}b^{1+y}\], где x+y=n \} 
\\
1.база: k=1 
$S\rightarrow aaKb$ верно\\
2.Пусть для n=k-1 выполняется 
\[ S\rightarrow   \{a^{k+1}Kb, a^{k+1}Kb^{k}, a^{k}b, a^{k}b^{k-1}, a^{2+x+y}Kb^{1+y}, a^{1+x+y}b^{y}, a^{1+x+y}b^{1+y}\], где x+y=k-1 \} \\
3. Тогда на k-ом шаге правило может осуществится только для элементов $\{a^{k+1}Kb, a^{k+1}Kb^{k}, a^{2+x+y}Kb^{1+y}$, где x+y=k-1 \} и будут получены элементы:\\
\[ \{a^{k+2}Kb, a^{2+k}Kb^{1+k}, a^{k+1}b, a^{1+k}b^{k}, a^{2+x+y}Kb^{1+y}, a^{1+x+y}b^{y}, a^{1+x+y}b^{1+y}\], где x+y=k \}
 
 
 \section*{Задача 3}
Из того, что $n\neq m \Leftrightarrow n < m $ or $ m < n \Rightarrow  $ . Опираемся на з.1\\
G: N = $\{S_1, S_2, A, B, S \}, T=\{a, b\}, S=S $\\ $P:$\\
\{
$ S\rightarrow S_1|S_2, S_1 \rightarrow aaAb, A\rightarrow aA|aAb|\epsilon, S_2 \rightarrow  aBbb, B\rightarrow Bb|aBb|\epsilon\}$\\

Доказательство аналогично док-ву из з.1\\
 
 \section*{Задача 4}
 $ a)(aa, bb, ab, ba)^* \\
 b) (b)^*(a)^*ab(b)^*(a)^* \\
 c) (a|b)^*(a|b)(A|B)(a|b)^*$ 
 \section*{Задача 5}
РВ = $txt|txtxt|txt(t|x)^*txt$ 

1. Очевидно, что слова, заданные РВ принадлежат языку, т.к. начинаются и оканчиваются на txt. \\
2. Всякое слово из языка начинается и оканчивается на txt, и если они не пересекаются, то между ними может содержать любую (в том чимсле пустую)комбинацию из x и t, что равно $(x|t)^*$, т.е. $\forall$ слово из языка может быть задано данным РВ.\\
Значит, РВ задано корректно.\\  
 
 \section*{Задача 6}
 
 У нас есть процедура генерации языка. Сохраним все слова языка,если он конечный или мы можем его сохранить. Далее, данное нам слово будем последовательно посимвольно сравнивать со словами нашего языка и при совпадении возврашать "да" или "нет",если слов конечное число и совпадения не было. Если же слов в языке не конечное число, то запустим процедуру генерации и каждое новое порожденное слово будем сравнивать со словом, принадлежность которого к языку изучается. В этом случае, кроме результат либо "да", либо процедура распознавания не остановится.\\ 
 
 Теперь пусть есть алгоритм распознавания. Создадим алгоритм генерации следующим образом: из упорядоченного по длинам слов, а для слов с равными длинами - в лексикографическом порядке - множества $ \{a, b\}^* $ будем последовательно подавать слова алгоритму распознавания, затем если он, возвращает результат "да", то данное слово возвращаем как сгенерированное.\\
\end{document}