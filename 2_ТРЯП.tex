\documentclass[a4paper,12pt]{article} % тип документа

% report, book

%  Русский язык

\usepackage[T2A]{fontenc}			% кодировка
\usepackage[utf8]{inputenc}			% кодировка исходного текста
\usepackage[english,russian]{babel}	% локализация и переносы


% Математика
\usepackage{amsmath,amsfonts,amssymb,amsthm,mathtools} 


\usepackage{wasysym}
% мои добьавки-----------------------------------------
\usepackage{ dsfont }
\usepackage{listings}

%мое----------------------------------------------------
%Заговолок
\author{Попов Николай}
\title{Домашнее задание 1\\ по алгебре \LaTeX{}}

%--------------------
\usepackage{tikz}












\begin{document} 

\section*{Задача 1}
При составлении правил учитываем, что длина слова - четное число и в каждой цепочке правил (1,2,3,4) нарушаем какое-либо свойство слов вычитаемого множества:\\

S=S,
$ T=\{a, b\},
 N = \{A, B, S, \epsilon\}\\
P = \{
S \rightarrow S_1|S_2|S_3|S_4,\\ S_1 \rightarrow \epsilon,\\ S_2 \rightarrow bbS_2|aaA|abA|baA, \\ A \rightarrow aaA|abA|baA|bbA|\epsilon,\\ S_3 \rightarrow aaS_3|\epsilon, \\ S_4 \rightarrow aaS_4|S_2
\}
$\\

1)пустое слово четной длины \\
2)слова (здесь и дальше, слова четной длины), начинающиеся с некоторого числа b, за коими следует $(a,b)^*$, содержащее хотя бы один литерал а.\\
3)слова только из a\\
4)слова из нескольких а, затем нескольких b, затем $ (a,b)^*$, содержащее хотя бы один литерал а.\\


\section*{Задача 2}
Пронумеруем правила:\\
$
S\rightarrow aSBC|aBC \hspace{15pt}(1)\\,
 CB\rightarrow BC, \hspace{15pt}(2)\\
 aB\rightarrow ab,\hspace{15pt}(3)\\
  bB\rightarrow bb,\hspace{15pt}(4)\\
   bC\rightarrow bc,\hspace{15pt}(5)\\
    cC\rightarrow cc \hspace{15pt}(6)$\\


    
Заметим, что в правилах пара сВ не преобразуется в терминальные символы, поэтому всякая цепочка, в которой три первые заглавных сивола после строчных это СBC,не будет окончательно преобразована в терминальные символы, а значит, не попадет в язык. Т.е. чтобы получить слово из языка, необходимо все литералы В повторением формулы 2 "переместить" после литералов а и до всех С.\\

Для преобразования строки $a^{k}(BC)^{k}$ к виду из терминальных символов необходимо повторить  операции (1) и (2), которые и будем доказывать.\\

(1) Докажем индукцией по числу букв а(т.е. и по числу пар ВС), что последовательность $a^{k}(BC)^{k}$ может быть приведена к виду $a^{k}B^{k}C^{k}$.\\

1)При k=1 получаем \{aBC\} - выполнено\\
2)Пусть это выполнено для k = n-1, $a^{n-1}(BC)^{n-1}$->$a^{n-1}B^{n-1}C^{n-1}$\\
3)Тогда при k=n $a^{n}(BC)^{n}=aa^{n-1}(BC)^{n-1}BC=aa^{n-1}B^{n-1}C^{n-1}BC=a^{n}B^{n}C^{n}$
после применения к цепочке СВ n-1 раз формулу 2

После получения вида $a^{n}B^{n}C^{n}$ аналогично, доказывается, что последовательность $a^{n}B^{n}C^{n}$ может быть приведена к виду $a^{n}b^{n}c^{n}$, причем, в третьем пункте доказательства применим последовательность формул 4, 5, 6\\
       


 



\section*{Задача 3}
Покажем, что нет ДКА, принимающего указанный язык. Из этого последует, что он нерегулярен.\\ 
Пусть такой ДКА есть. Т.к. в ДКА конечное число состояний k и |D(q, x)|$\leq$1 для x = a|b и $\forall q \in Q$ , то из к-ого состояния автомат перейдет в какое-то уже посещенное состояние i, а из него в i+1. Из этого состояния неободимо идти уже по ребру b в область автомата(т.е. подграф), принимающий $b^(k+1)$. Но если здесь есть ребро b, тогда такой автомат примет и слово $a^ib^(k+1)$, что невозможно, т.к. такое слово не принадлежит языку, распознаваемому автоматом. Значит, такого ребра нет и автомат остановится в неконечном состоянии i+1, т.е. слово принято не будет. Это противоречит тому, что слово принадлежит языку автомата. Значит, такого ДКА нет и указанный язык нерегулярен.\\  

(указаны ребра и состояния по которым идем)
\begin{center}
\begin{tikzpicture}[scale=0.2]
\tikzstyle{every node}+=[inner sep=0pt]
\draw [black] (8.3,-7.7) circle (3);
\draw (8.3,-7.7) node {$1$};
\draw [black] (21.7,-7.7) circle (3);
\draw (21.7,-7.7) node {$2$};
\draw [black] (40.3,-7.7) circle (3);
\draw (40.3,-7.7) node {$i$};
\draw [black] (76.2,-33.8) circle (3);
\draw (76.2,-33.8) node {$k$};
\draw [black] (76.2,-8.5) circle (3);
\draw (76.2,-8.5) node {$k-1$};
\draw [black] (52.1,-7.7) circle (3);
\draw (52.1,-7.7) node {$i+1$};
\draw [black] (13.3,-7.7) -- (18.7,-7.7);
\draw (12.8,-7.7) node [left] {$a$};
\fill [black] (18.7,-7.7) -- (17.9,-7.2) -- (17.9,-8.2);
\draw [black] (0,-7.7) -- (5.3,-7.7);
\fill [black] (5.3,-7.7) -- (4.5,-7.2) -- (4.5,-8.2);
\draw [black] (33.2,-7.7) -- (37.3,-7.7);
\draw (32.7,-7.7) node [left] {$a$};
\fill [black] (37.3,-7.7) -- (36.5,-7.2) -- (36.5,-8.2);
\draw [black] (76.2,-14) -- (76.2,-30.8);
\draw (76.2,-13.5) node [above] {$a$};
\fill [black] (76.2,-30.8) -- (76.7,-30) -- (75.7,-30);
\draw [black] (73.77,-32.04) -- (42.73,-9.46);
\fill [black] (42.73,-9.46) -- (43.08,-10.34) -- (43.67,-9.53);
\draw (59.35,-20.25) node [above] {$a$};
\draw [black] (45,-7.7) -- (49.1,-7.7);
\draw (44.5,-7.7) node [left] {$a$};
\fill [black] (49.1,-7.7) -- (48.3,-7.2) -- (48.3,-8.2);
\end{tikzpicture}
\end{center}

\section*{Задача 4}
Построим НКА, принимающий язык $L'$. Чтобы слово ух $\in L'$ принималось нашим НКА, необходимо изменить исходный ДКА так, чтобы первые два литерала теперь не считывались вначале, а считывались в конце. Для этого:\\
1) скопируем все цепочки ребер длины 2, исходящие из стартового состония ДКА(два первых ребра и вершина меж ними) и удалим их (вершину стартового состония ДКА удаляем). \\
2)Вершины, которые были на расстоянии 2 ребер от начальной в ДКА (а теперь остались концевыми) сделаем стартовыми в НКА.\\
3)Добавим новую вершину t и от всех конечных вершин ДКА (их было хотя бы одно) проведем ребра с пустым словом в эту вершину t, а из нее проведем все скопированные в п.1 пары последовательных ребер.\\
4) В конец каждой пары из этих ребер (п.3) добавим по вершине (удаленной на 2 ребра от начала в ДКА)точно с тем же состоянием, что было изначально в ДКА и сделаем их конечными.\\

Теперь первые два литерала слов из L будут считаны вконце. Осталось только с помощью запоминания начального состояния в НКА(k, k+1...m) сопоставить при чтении слова созданным НКА соответствующие друг другу цепочки х и у (*).\\

Пусть ДКА был описан:\\
$\{Q,T,D,q_0,F\}$ \\
Тогда опишем НКА:\\
$Q'$= $ Q \times Q $ , начальные состояния - $(q_n,q_n)$ - пары вершин удаленных на 2 ребра от начала в ДКА, Т тот же,правила:
$  D'(q_1,q_2,x)=(q_1,D(q_2,x))$,
конечные состояния - $q_n$, вершины удаленные на 2 ребра от начала в ДКА.\\
Сохраняемое состояние применим при попадании в конечную вершину для (*).\\

\begin{center}ДКА\end{center} 

\begin{center}
\begin{tikzpicture}[scale=0.2]
\tikzstyle{every node}+=[inner sep=0pt]
\draw [black] (5.5,-5.7) circle (3);
\draw (5.5,-5.7) node {$S$};
\draw [black] (21.4,-6.4) circle (3);
\draw (21.4,-6.4) node {$2$};
\draw [black] (17.8,-14.3) circle (3);
\draw (17.8,-14.3) node {$3$};
\draw [black] (5.5,-21.8) circle (3);
\draw (5.5,-21.8) node {$n$};
\draw [black] (35.5,-9.1) circle (3);
\draw (35.5,-9.1) node {$k$};
\draw [black] (28.1,-21.2) circle (3);
\draw (28.1,-21.2) node {$k+1$};
\draw [black] (16.1,-29.6) circle (3);
\draw (16.1,-29.6) node {$m$};
\draw [black] (63.3,-10.9) circle (3);
\draw (63.3,-10.9) node {$F_1$};
\draw [black] (63.3,-10.9) circle (2.4);
\draw [black] (58,-25) circle (3);
\draw (58,-25) node {$F_2$};
\draw [black] (58,-25) circle (2.4);
\draw [black] (54.5,-37.7) circle (3);
\draw (54.5,-37.7) node {$F_n$};
\draw [black] (54.5,-37.7) circle (2.4);
\draw [black] (-0.3,-5.7) -- (2.5,-5.7);
\fill [black] (2.5,-5.7) -- (1.7,-5.2) -- (1.7,-6.2);
\draw [black] (7.909,-3.928) arc (118.48238:56.47597:10.929);
\fill [black] (19.16,-4.42) -- (18.77,-3.56) -- (18.21,-4.4);
\draw (13.74,-1.97) node [above] {$x[1],1$};
\draw [black] (7.96,-7.42) -- (15.34,-12.58);
\fill [black] (15.34,-12.58) -- (14.97,-11.71) -- (14.4,-12.53);
\draw (8.2,-10.5) node [below] {$x[1],2$};
\draw [black] (5.5,-8.7) -- (5.5,-18.8);
\fill [black] (5.5,-18.8) -- (6,-18) -- (5,-18);
\draw (5,-13.75) node [left] {$x[1],n$};
\draw [black] (24.35,-6.96) -- (32.55,-8.54);
\fill [black] (32.55,-8.54) -- (31.86,-7.89) -- (31.67,-8.88);
\draw (26.62,-8.56) node [below] {$x[2].1$};
\draw [black] (20.29,-15.97) -- (25.61,-19.53);
\fill [black] (25.61,-19.53) -- (25.22,-18.67) -- (24.66,-19.5);
\draw (19.5,-18.25) node [below] {$x[2],2$};
\draw [black] (7.92,-23.58) -- (13.68,-27.82);
\fill [black] (13.68,-27.82) -- (13.34,-26.95) -- (12.74,-27.75);
\draw (7.35,-26.2) node [below] {$x[2],n$};
\end{tikzpicture}
\end{center} 

Конечных состояний $F_i$ в ДКА хотя бы одно. Меж конечными состояниями и состояниями после прочтения х находится ненарисованная часть ДКА, в которой считывается у.\\   

\begin{center}НКА\end{center}


\begin{center}
\begin{tikzpicture}[scale=0.2]
\tikzstyle{every node}+=[inner sep=0pt]
\draw [black] (29.5,-5.3) circle (3);
\draw (29.5,-5.3) node {$k$};
\draw [black] (23.6,-11.5) circle (3);
\draw (23.6,-11.5) node {$k+1$};
\draw [black] (14,-22.6) circle (3);
\draw (14,-22.6) node {$m$};
\draw [black] (40,-9.6) circle (3);
\draw (40,-9.6) node {$F_1$};
\draw [black] (35.3,-17.8) circle (3);
\draw (35.3,-17.8) node {$F_2$};
\draw [black] (23.6,-30.5) circle (3);
\draw (23.6,-30.5) node {$F_n$};
\draw [black] (43.5,-29.1) circle (3);
\draw (43.5,-29.1) node {$t$};
\draw [black] (63.3,-22.6) circle (3);
\draw (63.3,-22.6) node {$2$};
\draw [black] (46.3,-45) circle (3);
\draw (46.3,-45) node {$n$};
\draw [black] (58.3,-29.8) circle (3);
\draw (58.3,-29.8) node {$3$};
\draw [black] (75.5,-30.5) circle (3);
\draw (75.5,-30.5) node {$k$};
\draw [black] (75.5,-30.5) circle (2.4);
\draw [black] (69.9,-38.1) circle (3);
\draw (69.9,-38.1) node {$k+1$};
\draw [black] (69.9,-38.1) circle (2.4);
\draw [black] (58.3,-52.8) circle (3);
\draw (58.3,-52.8) node {$m$};
\draw [black] (58.3,-52.8) circle (2.4);
\draw [black] (40.53,-12.55) -- (42.97,-26.15);
\fill [black] (42.97,-26.15) -- (43.32,-25.27) -- (42.34,-25.45);
\draw (41.03,-19.61) node [left] {$e$};
\draw [black] (37.06,-20.23) -- (41.74,-26.67);
\fill [black] (41.74,-26.67) -- (41.67,-25.73) -- (40.86,-26.32);
\draw (38.81,-24.83) node [left] {$e$};
\draw [black] (26.59,-30.29) -- (40.51,-29.31);
\fill [black] (40.51,-29.31) -- (39.67,-28.87) -- (39.74,-29.87);
\draw (33.66,-30.37) node [below] {$e$};
\draw [black] (46.183,-27.758) arc (115.16264:101.1856:61.249);
\fill [black] (60.34,-23.11) -- (59.46,-22.77) -- (59.66,-23.76);
\draw (50.07,-24.37) node [above] {$x[1],1$};
\draw [black] (46.5,-29.24) -- (55.3,-29.66);
\fill [black] (55.3,-29.66) -- (54.53,-29.12) -- (54.48,-30.12);
\draw (50.74,-30.11) node [below] {$x[1],2$};
\draw [black] (44.02,-32.05) -- (45.78,-42.05);
\fill [black] (45.78,-42.05) -- (46.13,-41.17) -- (45.15,-41.34);
\draw (44.18,-37.3) node [left] {$x[1],n$};
\draw [black] (66.053,-23.789) arc (64.1762:49.97438:34.864);
\fill [black] (73.29,-28.47) -- (73,-27.58) -- (72.35,-28.34);
\draw (73.27,-25.41) node [above] {$x[2],1$};
\draw [black] (60.74,-31.55) -- (67.46,-36.35);
\fill [black] (67.46,-36.35) -- (67.1,-35.48) -- (66.52,-36.3);
\draw (67.55,-33.45) node [above] {$x[2],2$};
\draw [black] (48.82,-46.63) -- (55.78,-51.17);
\fill [black] (55.78,-51.17) -- (55.39,-50.31) -- (54.84,-51.15);
\draw (48.85,-49.4) node [below] {$x[2],n$};
\draw [black] (7.2,-18.1) -- (11.5,-20.94);
\fill [black] (11.5,-20.94) -- (11.11,-20.09) -- (10.56,-20.92);
\draw [black] (16.1,-7.6) -- (20.94,-10.12);
\fill [black] (20.94,-10.12) -- (20.46,-9.3) -- (20,-10.19);
\draw [black] (20.9,-1.8) -- (26.72,-4.17);
\fill [black] (26.72,-4.17) -- (26.17,-3.4) -- (25.79,-4.33);
\end{tikzpicture}
\end{center}

 Меж состояниями $k, k+1 ... m$  и состояниями $ F_i $ находится ненарисованная часть из ДКА, в которой считывается у.\\  


\section*{Задача 5}
Аналогично примеру на семинаре:\\
\begin{center}
\begin{tikzpicture}[scale=0.2]
\tikzstyle{every node}+=[inner sep=0pt]
\draw [black] (5.9,-3.7) circle (3);
\draw (5.9,-3.7) node {$1$};
\draw [black] (15.8,-7.1) circle (3);
\draw (15.8,-7.1) node {$2$};
\draw [black] (26.7,-9.3) circle (3);
\draw (26.7,-9.3) node {$3$};
\draw [black] (35,-3.7) circle (3);
\draw (35,-3.7) node {$9$};
\draw [black] (40,-13.4) circle (3);
\draw (40,-13.4) node {$6$};
\draw [black] (47.2,-4.2) circle (3);
\draw (47.2,-4.2) node {$4$};
\draw [black] (59.7,-4.2) circle (3);
\draw (59.7,-4.2) node {$5$};
\draw [black] (72.8,-4.8) circle (3);
\draw (72.8,-4.8) node {$10$};
\draw [black] (51.8,-13.4) circle (3);
\draw (51.8,-13.4) node {$7$};
\draw [black] (62.7,-14.1) circle (3);
\draw (62.7,-14.1) node {$8$};
\draw [black] (74.7,-23.2) circle (3);
\draw (74.7,-23.2) node {$11$};
\draw [black] (33.6,-23.2) circle (3);
\draw (33.6,-23.2) node {$12$};
\draw [black] (5.9,-21.4) circle (3);
\draw (5.9,-21.4) node {$13$};
\draw [black] (12.5,-40.5) circle (3);
\draw (12.5,-40.5) node {$15$};
\draw [black] (9.4,-31.3) circle (3);
\draw (9.4,-31.3) node {$14$};
\draw [black] (5.9,-52.4) circle (3);
\draw (5.9,-52.4) node {$16$};
\draw [black] (17.4,-52.4) circle (3);
\draw (17.4,-52.4) node {$17$};
\draw [black] (30.7,-52.1) circle (3);
\draw (30.7,-52.1) node {$18$};
\draw [black] (40,-52.1) circle (3);
\draw (40,-52.1) node {$19$};
\draw [black] (56.1,-48) circle (3);
\draw (56.1,-48) node {$20$};
\draw [black] (28.3,-32.3) circle (3);
\draw (28.3,-32.3) node {$21$};
\draw [black] (40,-38.9) circle (3);
\draw (40,-38.9) node {$22$};
\draw [black] (72.8,-51.5) circle (3);
\draw (72.8,-51.5) node {$23$};
\draw [black] (72.8,-51.5) circle (2.4);
\draw [black] (-2.4,-3.7) -- (2.9,-3.7);
\fill [black] (2.9,-3.7) -- (2.1,-3.2) -- (2.1,-4.2);
\draw [black] (8.643,-2.496) arc (108.12269:41.74034:15.408);
\fill [black] (24.93,-6.88) -- (24.77,-5.95) -- (24.03,-6.62);
\draw (18.29,-1.69) node [above] {$e$};
\draw [black] (36.37,-6.37) -- (38.63,-10.73);
\fill [black] (38.63,-10.73) -- (38.7,-9.79) -- (37.81,-10.25);
\draw (36.81,-9.69) node [left] {$e$};
\draw [black] (8.74,-4.67) -- (12.96,-6.13);
\fill [black] (12.96,-6.13) -- (12.37,-5.39) -- (12.04,-6.34);
\draw (9.86,-5.94) node [below] {$a$};
\draw [black] (24.605,-11.436) arc (-51.8433:-158.29368:11.614);
\fill [black] (6.64,-6.6) -- (6.47,-7.53) -- (7.4,-7.16);
\draw (13.56,-14.09) node [below] {$e$};
\draw [black] (18.74,-7.69) -- (23.76,-8.71);
\fill [black] (23.76,-8.71) -- (23.07,-8.06) -- (22.88,-9.04);
\draw (20.63,-8.8) node [below] {$e$};
\draw [black] (29.19,-7.62) -- (32.51,-5.38);
\fill [black] (32.51,-5.38) -- (31.57,-5.41) -- (32.13,-6.24);
\draw (31.95,-7) node [below] {$e$};
\draw [black] (38,-3.82) -- (44.2,-4.08);
\fill [black] (44.2,-4.08) -- (43.42,-3.54) -- (43.38,-4.54);
\draw (41.06,-4.49) node [below] {$e$};
\draw [black] (50.2,-4.2) -- (56.7,-4.2);
\fill [black] (56.7,-4.2) -- (55.9,-3.7) -- (55.9,-4.7);
\draw (53.45,-4.7) node [below] {$b$};
\draw [black] (62.7,-4.34) -- (69.8,-4.66);
\fill [black] (69.8,-4.66) -- (69.03,-4.13) -- (68.98,-5.13);
\draw (66.2,-5.05) node [below] {$e$};
\draw [black] (43,-13.4) -- (48.8,-13.4);
\fill [black] (48.8,-13.4) -- (48,-12.9) -- (48,-13.9);
\draw (45.9,-13.9) node [below] {$b$};
\draw [black] (54.79,-13.59) -- (59.71,-13.91);
\fill [black] (59.71,-13.91) -- (58.94,-13.36) -- (58.88,-14.36);
\draw (57.15,-14.32) node [below] {$b$};
\draw [black] (64.91,-12.07) -- (70.59,-6.83);
\fill [black] (70.59,-6.83) -- (69.67,-7.01) -- (70.34,-7.74);
\draw (68.87,-9.94) node [below] {$e$};
\draw [black] (73.11,-7.78) -- (74.39,-20.22);
\fill [black] (74.39,-20.22) -- (74.81,-19.37) -- (73.81,-19.47);
\draw (73.1,-14.09) node [left] {$e$};
\draw [black] (71.704,-23.364) arc (-87.1173:-92.8827:349.055);
\fill [black] (36.6,-23.36) -- (37.37,-23.9) -- (37.42,-22.9);
\draw (54.15,-24.31) node [below] {$e$};
\draw [black] (8.862,-20.925) arc (98.63683:78.36582:178.792);
\fill [black] (8.86,-20.92) -- (9.73,-21.3) -- (9.58,-20.31);
\draw (40.4,-18.43) node [above] {$e$};
\draw [black] (6.9,-24.23) -- (8.4,-28.47);
\fill [black] (8.4,-28.47) -- (8.6,-27.55) -- (7.66,-27.88);
\draw (8.41,-25.59) node [right] {$a$};
\draw [black] (10.36,-34.14) -- (11.54,-37.66);
\fill [black] (11.54,-37.66) -- (11.76,-36.74) -- (10.81,-37.06);
\draw (11.72,-35.19) node [right] {$e$};
\draw [black] (9.637,-39.634) arc (-114.31362:-207.56122:11.488);
\fill [black] (4.18,-23.85) -- (3.37,-24.33) -- (4.26,-24.79);
\draw (2.75,-33.65) node [left] {$e$};
\draw [black] (8.832,-21.99) arc (70.77386:-32.64871:10.943);
\fill [black] (14.44,-38.23) -- (15.29,-37.82) -- (14.45,-37.28);
\draw (16.33,-28.01) node [right] {$e$};
\draw [black] (11.04,-43.12) -- (7.36,-49.78);
\fill [black] (7.36,-49.78) -- (8.18,-49.32) -- (7.31,-48.83);
\draw (8.53,-45.25) node [left] {$b$};
\draw [black] (8.9,-52.4) -- (14.4,-52.4);
\fill [black] (14.4,-52.4) -- (13.6,-51.9) -- (13.6,-52.9);
\draw (11.65,-52.9) node [below] {$e$};
\draw [black] (37.559,-53.838) arc (-59.38264:-119.09632:17.752);
\fill [black] (37.56,-53.84) -- (36.62,-53.82) -- (37.13,-54.68);
\draw (28.76,-56.83) node [below] {$e$};
\draw [black] (19.677,-50.454) arc (125.12502:56.39602:15.941);
\fill [black] (19.68,-50.45) -- (20.62,-50.4) -- (20.04,-49.58);
\draw (28.63,-47.04) node [above] {$e$};
\draw [black] (20.4,-52.33) -- (27.7,-52.17);
\fill [black] (27.7,-52.17) -- (26.89,-51.69) -- (26.91,-52.69);
\draw (24.06,-52.78) node [below] {$b$};
\draw [black] (33.7,-52.1) -- (37,-52.1);
\fill [black] (37,-52.1) -- (36.2,-51.6) -- (36.2,-52.6);
\draw (35.35,-51.6) node [above] {$e$};
\draw [black] (42.91,-51.36) -- (53.19,-48.74);
\fill [black] (53.19,-48.74) -- (52.29,-48.45) -- (52.54,-49.42);
\draw (48.86,-50.62) node [below] {$e$};
\draw [black] (32.09,-25.79) -- (29.81,-29.71);
\fill [black] (29.81,-29.71) -- (30.64,-29.27) -- (29.78,-28.76);
\draw (30.3,-26.51) node [left] {$b$};
\draw [black] (34.73,-25.98) -- (38.87,-36.12);
\fill [black] (38.87,-36.12) -- (39.03,-35.19) -- (38.1,-35.57);
\draw (36.06,-31.95) node [left] {$e$};
\draw [black] (36.14,-24.783) arc (51.17388:-6.81801:12.485);
\fill [black] (36.14,-24.78) -- (36.45,-25.67) -- (37.08,-24.9);
\draw (40.62,-28.89) node [right] {$e$};
\draw [black] (30.91,-33.77) -- (37.39,-37.43);
\fill [black] (37.39,-37.43) -- (36.94,-36.6) -- (36.44,-37.47);
\draw (33.05,-36.1) node [below] {$e$};
\draw [black] (42.61,-40.38) -- (53.49,-46.52);
\fill [black] (53.49,-46.52) -- (53.04,-45.69) -- (52.55,-46.57);
\draw (46.95,-43.95) node [below] {$e$};
\draw [black] (59.04,-48.62) -- (69.86,-50.88);
\fill [black] (69.86,-50.88) -- (69.18,-50.23) -- (68.98,-51.21);
\draw (63.8,-50.34) node [below] {$e$};
\draw [black] (71.382,-48.858) arc (-155.30954:-212.37236:24.372);
\fill [black] (71.38,-48.86) -- (71.5,-47.92) -- (70.59,-48.34);
\draw (68.61,-37.01) node [left] {$e$};
\draw [black] (73,-48.51) -- (74.5,-26.19);
\fill [black] (74.5,-26.19) -- (73.95,-26.96) -- (74.94,-27.02);
\draw (74.35,-37.39) node [right] {$e$};
\end{tikzpicture}
\end{center}


\end{document}